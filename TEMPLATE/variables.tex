\DeclareMathOperator{\sech}{sech}
\DeclareMathOperator{\csch}{csch}
\DeclareMathOperator{\arcsec}{arcsec}
\DeclareMathOperator{\arccot}{arcCot}
\DeclareMathOperator{\arccsc}{arcCsc}
\DeclareMathOperator{\arccosh}{arcCosh}
\DeclareMathOperator{\arcsinh}{arcsinh}
\DeclareMathOperator{\arctanh}{arctanh}
\DeclareMathOperator{\arcsech}{arcsech}
\DeclareMathOperator{\arccsch}{arcCsch}
\DeclareMathOperator{\arccoth}{arcCoth} 
\DeclareMathOperator{\grad}{\vv{\text{grad}}} 
\DeclareMathOperator{\conj}{^*} 
\DeclareMathOperator{\vect}{vv} 
\DeclareMathOperator{\Vect}{\text{Vect}} 
\DeclareMathOperator{\degree}{c^\circ} 
\DeclareMathOperator{\degre}{^\circ} 
\DeclareMathOperator{\transpose}{^\dagger} 
\DeclareMathOperator{\adjoint}{^\dagger} 

\newcommand{\moyenne}[1]{\langle #1 \rangle} 
\newcommand{\lagrange}{\mathcal{L}}
\newcommand{\fourier}{\mathcal{F}}
\newcommand{\hilbert}{\mathcal{H}}
\newcommand{\p}{\mathcal{P}}
\newcommand{\x}{\chi}
\newcommand{\ve}[1]{\vv{#1}}
\newcommand{\push}[1]{\begin{adjustwidth}{5mm}{}#1\end{adjustwidth}}
\newcommand{\operator}[1]{\widehat{#1}}
\newcommand{\HRule}{\rule{\linewidth}{0.5mm}} % Defines a new command for the horizontal lines, change thickness here
\renewcommand{\chaptermark}[1]{\markboth{\MakeUppercase{#1}}{}}
\renewcommand{\headrule}{%
\vspace{-8pt}\hrulefill
\raisebox{-2.1pt}{\quad\decofourleft\decotwo\decofourright\quad}\hrulefill}
\definecolor{myred}{RGB}{255, 14, 0}
\everymath{\displaystyle}

\def\changemargin#1{\list{}{\leftmargin#1}\item[]}
\let\endchangemargin=\endlist 

\makeatletter
\newcommand*{\rom}[1]{\expandafter\@slowromancap\romannumeral #1@}%roman numbers
\makeatother

%hyperlink shit

\hypersetup{
    colorlinks,
    citecolor=black,
    filecolor=black,
    linkcolor=black,
    urlcolor=black
}
% Table specail cell , it's for making line break in table cell
\newcommand{\specialcell}[2][c]{%
  \begin{tabular}[#1]{@{}c@{}}#2\end{tabular}}

  \definecolor{main}{HTML}{5989cf}    % setting main color to be used
\definecolor{sub}{HTML}{cde4ff}     % setting sub color to be used

\tcbset{
    sharp corners,
    colback = white,
    before skip = 0.2cm,    % add extra space before the box
    after skip = 0.5cm      % add extra space after the box
}   
\newtcolorbox{boxH}{
    colback = sub, 
    colframe = main, 
    boxrule = 0pt, 
    leftrule = 6pt % left rule weight
}
\newtcolorbox{gpt}{
    sharpish corners, % better drop shadow
    boxrule = 0pt,
    toprule = 4.5pt, % top rule weight
    enhanced,
    fuzzy shadow = {0pt}{-2pt}{-0.5pt}{0.5pt}{black!35} % {xshift}{yshift}{offset}{step}{options} 
}
\definecolor{codegreen}{rgb}{0,0.6,0}
\definecolor{codegray}{rgb}{0.5,0.5,0.5}
\definecolor{codepurple}{rgb}{0.58,0,0.82}
\definecolor{backcolour}{rgb}{0.95,0.95,0.92}

\lstdefinestyle{mystyle}{
    backgroundcolor=\color{backcolour},   
    commentstyle=\color{codegreen},
    keywordstyle=\color{magenta},
    numberstyle=\tiny\color{codegray},
    stringstyle=\color{codepurple},
    basicstyle=\ttfamily\footnotesize,
    breakatwhitespace=false,         
    breaklines=true,                 
    captionpos=b,                    
    keepspaces=true,                 
    numbers=left,                    
    numbersep=5pt,                  
    showspaces=false,                
    showstringspaces=false,
    showtabs=false,                  
    tabsize=2,
    basicstyle = \small
}

\lstset{style=mystyle}